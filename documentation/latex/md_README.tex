O objetivo deste projeto é criar um ambiente para processamento dos dados que são entregues pelo sistema PETSys TOFPET2, da PETSys® Electronics. Este será fito de maneira modularizada, com funções que permitem o calculo do posicionamento dos eventos, estimativa do tempo do evento, além de correções necessárias para a utilização do mesmo.

O projeto está organizado da seguinte forma \+:


\begin{DoxyItemize}
\item ./base \+: Local onde se encontram os codigos de base para o processamento, responsáveis pela escrita de arquivos, montagem de histogramas entre outros
\item ./caracterização\+: Local onde serão armazenados os codigos específicos a serem utilizados para a caracterização do sistema
\item ./outout\+: Local onde serão salvos os arquivos processados
\item ./\+Run.C \+: Arquivo que deve ser executado para carregar o projeto no ROOT
\end{DoxyItemize}

Para carregar as funções e classes implementadas ao root, navegue ate a pasta onde o projeto foi baixado e abra o root, logo no terminal do root digite\+:

\textquotesingle{}\textquotesingle{}\textquotesingle{}

.X \mbox{\hyperlink{Run_8C}{Run.\+C}}

\textquotesingle{}\textquotesingle{}\textquotesingle{}

Isso irá carregar todas as funções e classes implementadas para a memoria, sendo facilmente utilizadas no terminal do ROOT. 